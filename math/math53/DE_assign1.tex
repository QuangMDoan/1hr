%\documentclass{article}

\documentclass[11pt]{article}

\usepackage{amsmath}
\usepackage{amsfonts}
\usepackage{amssymb}
\usepackage{mathtools} 
\usepackage{amsthm}
\usepackage[dvipsnames]{xcolor}
\usepackage{setspace}
\usepackage{enumitem}
\usepackage{romannum}
\usepackage{comment}

\usepackage{xfp}
\usepackage{fp}
\usepackage{forloop}
\usepackage{pgffor}

\makeatletter % Use \makeatletter to access internal LaTeX commands

%% Sets margins
\usepackage{geometry}
 \geometry{
 letterpaper,
 total={165mm,257mm},
 left=20mm,
 top=20mm,
 }
 
 

%%NEWENVIRONMENTS:
\newenvironment{ea}{\begin{list}{\alph{ea}.}
	{\usecounter{ea}}}{\end{list}}

%NEWCOUNTERS:
\newcounter{ea}

%NEW COMMANDS
\newcommand {\Problem} [1] { \bigskip \noindent\Large {\bf Problem #1} \\  \normalsize}
\newcommand {\myLine} {\bigskip \noindent\rule{17.7cm}{0.4pt} \medskip}

\newcommand {\mySpaceLeft}{\the\dimexpr\pagegoal-\pagetotal\relax}


\newcommand{\fillSpaceWithLines}[1]{

  \edef\mySpaceLeft{\the\dimexpr\pagegoal-\pagetotal\relax}
  \edef\myReal{\fpeval{\mySpaceLeft/1pt}}
  \edef\myLines{\fpeval{round(\myReal/#1)}}
  \foreach \varLines in {1,...,\myLines} {
    \dotfill
    \vspace{3mm}
    
  }
  \eject
}  

\newcommand{\fillWithLines}[1]{

\bigskip


\bigskip

	\fillSpaceWithLines {#1}
}

\newcommand{\addPage}{

Additional sheet for problem \rule{3cm}{0.2pt}
\bigskip

\fillSpaceWithLines {26}

}

%\excludecomment{latex}
\includecomment{latex}
\excludecomment{handout}
%\includecomment{handout}

\begin{handout}
\newenvironment{Solution}{
	\bigskip

	\fillSpaceWithLines {26}
	}{ }
\end{handout}

\begin{latex}
\newenvironment{Solution}{
	\underline{Solution:}
	}{ }
\end{latex}

	
%% Initial instructions: 
%% Below is the "visible" part of the LaTeX file. Include your solution where it is marked.
%% Note that some *new commands* have been defined above. You may use them instead of the original LaTeX commands. 
%% Suggesting to use the $$...$$ to display centered text of formulas to emphasize your major results.
%% Please keep one problem per page for easier reading.

\begin{document}

\setstretch {1.2}

\begin{center}
  \begin{Large}
    XM531 Problem Set 1
    
    \medskip
    Quang Doan
  \end{Large} \\
\end{center}

\medskip


%%%%%%%%%%%%%%%%%%

{\color {Blue} 

\Problem 1 For each of the following give an example. You do not need to solve your example.

\begin{enumerate}[label=(\alph*)]
    
\item  Give an example of a second-order ODE that is non-linear.

\item  Give an example of a first-order ODE that is linear but non-separable.

\item  Give an example of a first-order ODE that is separable but non-linear.

\item  Give an example of a first-order ODE that is non-separable and non-linear or explain why such a differential equation cannot exist.

\item  Give an example of a first-order ODE that is separable and linear or explain why such a differential equation cannot exist.
\end{enumerate}

\myLine

}

\begin{Solution}

\begin{enumerate}[label=(\alph*)]   
\item  Give an example of a second-order ODE that is non-linear.
$$ y^{''} + y^{'} + y^2 = 0 $$
\item  Give an example of a first-order ODE that is linear but non-separable.
$$ y^{'} + x^2y = x $$
\item  Give an example of a first-order ODE that is separable but non-linear.
$$ \frac{dy}{dx} + xy^2 = 0 $$

\item  Give an example of a first-order ODE that is non-separable and non-linear or explain why such a differential equation cannot exist.
$$ \frac{dy}{dx} + e^y = x $$

\item  Give an example of a first-order ODE that is separable and linear or explain why such a differential equation cannot exist.
$$ \frac{dy}{dx} + y = 0 $$

\end{enumerate}

\end{Solution}

\eject

%%%%%%%%%%%%%%%%%%

{\color {Blue} 

\Problem 2 Sometimes it is possible to solve a nonlinear equation by making a change of the dependent variable that converts it into a linear equation. The most important equation has the form

$$y' + p(t)y = q(t)y^n,$$

and is called a Bernoulli equation after Jakob Bernoulli. 

\begin{enumerate}[label=(\alph*)]
\item  Solve the Bernoulli equation when $n=0$; when $n=1$.

\item  Show that if $n\neq 0,1,$ then the substitution $v = y^{1-n}$ reduces Bernoulli's equation to a linear equation. This method of solution was found by Leibniz in 1696.

\end{enumerate}

\myLine
}


\begin{Solution}

\begin{enumerate}[label=(\alph*)]
\item  Solve the Bernoulli Equation when $n=0$; when $n=1$.

For $n = 0,$ Solve the Bernoulli Equation  $y' + p(t)y = q(t) $

Times the Bernoulli Equation by integrating factor: $ e^{\int{p(t) dt}}$

$$  y'e^{\int{p(t) dt}} + yp(t) e^{\int{p(t) dt}} = q(t) e^{\int{p(t) dt}}  $$
$$  \frac{d}{dt}({y e^{\int{p(t) dt}}}) = q(t) e^{\int{p(t) dt}} $$
$$ \int{\frac{d}{dt}({y e^{\int{p(t) dt}}})} dt = \int{q(t) e^{\int{p(t) dt}}} dt $$
$$ y e^{\int{p(t) dt}} = \int{q(t) e^{\int{p(t) dt}}} dt $$
$$ y = \frac{1}{e^{\int{p(t) dt}}} \int{q(t) e^{\int{p(t) dt}}} dt $$


For $n = 1,$ Solve the Bernoulli Equation  $y' + p(t)y = q(t)y $

$$  \frac{dy}{dt} =( q(t) - p(t) ) y $$
$$  \frac{dy}{y} =( q(t) - p(t) ) dt $$
$$  \int \frac{dy}{y} =\int ( q(t) - p(t) ) dt $$
$$  \ln{|y|} =\int ( q(t) - p(t) ) dt $$
$$  y =e^ {\int ( q(t) - p(t) ) dt} $$

\eject


\item  Show that if $n\neq 0,1,$ then the substitution $v = y^{1-n}$ reduces Bernoulli's equation to a linear equation. This method of solution was found by Leibniz in 1696.

$$ y' + p(t)y = q(t) y^n $$
$$ y' = q(t) y^n - p(t)y $$
$$ \frac{y'}{y^n}  = q(t) - p(t)y^{1-n}    $$ 

Let $v = y^{1-n} $
$$ v' = (1-n)\frac{y'}{y^n} $$
$$ y' = \frac{y^n}{1-n} v'$$

$$ \frac{y'}{y^n}  = q(t) - p(t)y^{1-n}  $$ 
$$ \frac{y^n}{(1-n)*y^n} v'  = q(t) - p(t)v    $$
$$ \frac{v'}{(1-n)}  = q(t) - p(t)v  $$
$$ v'  = (1-n)q(t) - (1-n)p(t)v  $$

so if $n\neq 0,1,$ then the substitution $v = y^{1-n}$ reduces Bernoulli's equation to a linear ODE.

\end{enumerate}

\end{Solution}

\eject

%%%%%%%%%%%%%%%%%%

{\color {Blue} 

\Problem 3 Find a substitution of $v$ that reduces the problem
$$y' = f(ax+by+c),$$
into a separable differential equation.

\myLine
}


\begin{Solution}

$$ \frac{dy}{dx} = f(ax+by+c) $$

Let $ v = ax+by+c $

$$ \frac{dv}{dx} = a +b \frac{dy}{dx} + 0 $$

$$ \frac{dv}{dx} = a +b f(v) $$

$$ \frac{dv}{a +b f(v)} = dx $$ 

$ \frac{dv}{a +b f(v)} = dx $ is a separable differential equation

\end{Solution}

\eject

%%%%%%%%%%%%%%%%%%

{\color {Blue} 
\Problem 4
Using an appropriate substitution, find the general solution to the ODEs.
\begin{enumerate}[label=(\alph*)]
\item  $\displaystyle{y' = (2x+3y)^2}$.
\item  $y^2y'+\dfrac{y^3}{t}=\dfrac{2}{t^2}$.
\item  $(x^2-y^2)dx+xydy=0$.
\end{enumerate}
\myLine
}


\begin{Solution}

\begin{enumerate}[label=(\alph*)]
\item Find the general solution to $ y' = (2x+3y)^2 $.
$$ \frac{dy}{dx} = (2x+3y)^2 $$

Let $ v = 2x+3y $
$$ \frac{dv}{dx} =  2 +  3 \frac{dy}{dx} $$
$$ \frac{dv}{dx} =  2 +  3 v^2 $$
$$ \frac{dv}{2 +  3 v^2} =  dx $$

Integrate both sides:
$$ \int \frac{dv}{2 +  3 v^2} = \int  dx $$
$$ \dfrac{1}{\sqrt6} \tan^{-1}(\sqrt{\dfrac{3}{2}} v) = x + C $$

Re-substitute back $ v = 2x+3y $ to have implicit solution
$$ \tan^{-1}(\sqrt{\dfrac{3}{2}} (2x+3y)) = \sqrt6x + C $$

\eject

\item Find the general solution to $ y^2y'+\dfrac{y^3}{t}=\dfrac{2}{t^2} $.

$$y^2y'+\dfrac{y^3}{t} = \dfrac{2}{t^2} $$

Divide both sides by $ y^2 $ yields Bernoulli equation  

$$ y'+ \dfrac{1}{t}y = \dfrac{2}{t^2}y^{-2} $$

Let $ v = y^{1-n}, v = y^3 $ 

$$v' =3y^2 y' $$
$$y' = \frac{1}{3y^2} v' $$

Substitute $ v = y^3, y' = \frac{1}{3y^2} v' $ to original equation $ y^2y'+\dfrac{y^3}{t}=\dfrac{2}{t^2} $ 

$$ \frac{y^2}{3y^2} v' + \dfrac{1}{t}v = \dfrac{2}{t^2} $$
$$ v' + \dfrac{3}{t}v = \dfrac{6}{t^2} $$

Multiple both sides by integrating factor $ e^{\int{\frac{3}{t} dt}} = t^3$
$$ t^3v' + 3t^2v = 6t $$

$$ \frac{d}{dt}(t^3v) = 6t $$

$$ \int \frac{d}{dt}(t^3v) dt = \int 6t dt $$

$$ t^3v = 3t^2 + C $$

$$ v = \frac{3}{t} + \frac{C}{t^3}  =  \frac{3t^2 + C}{t^3}  $$

Re-substitute $ v = y^3, y = v^{1/3} $

$$ y = \frac{(3t^2 + C)^{1/3}}{t} $$

\eject

\item Find the general solution to $(x^2-y^2)dx+xydy=0$.
$$ (x^2-y^2)dx+xydy=0 $$
Divide both sides by $dx$
$$ (x^2-y^2)+xyy'=0 $$
Divide both sides by $ xy $
$$ \frac{x^2}{xy} - \frac{y^2}{xy} + y'=0 $$
$$ y' - \frac{1}{x} y = -x y^{-1} $$
The DE is the Bernoulli's equation has the form $y' + p(x)y = q(x)y^n, n = -1 $
$$ y' - \frac{1}{x} y = -x y^{-1} $$
Multiply both sides by $2y$
$$ 2yy' - (2y)\frac{1}{x} y = (2y) -x y^{-1} $$
$$ 2yy' - \frac{2}{x} y^2 = -2x $$
Substitute $ v = y^{1-n} = y^2, v' = 2yy', y' = \frac{1}{2y} v'$
$$ v' - \frac{2}{x} v = -2x $$
Multiply both sides by integrating factor $e^{-2\int{ \frac{1}{x}} dx } = \frac{1}{x^2} $
$$ \frac{1}{x^2}v' - \frac{1}{x^2} \frac{2}{x} v = -2x \frac{1}{x^2}$$
$$  \frac{d}{dx}(\frac{1}{x^2}v) = -\frac{2}{x} $$ 
$$ \int \frac{d}{dx}(\frac{1}{x^2}v) dx = \int -\frac{2}{x} dx $$ 
$$ \frac{v}{x^2} = -2 \ln{x} + C $$
$$ v = -2 x^2 \ln{x} + Cx^2 $$
$$ v = x^2 (C-2\ln{x})$$
Re-substitute $ v = y^2 $ into $ v = x^2 (C-2\ln{x})$
$$ y^2 = x^2 (C-2\ln{x}) $$
$$ y = x \sqrt{C-2 \ln{x}} $$
or 
$$ y = -x \sqrt{C-2 \ln{x}} $$
\end{enumerate}
\end{Solution}

\eject

%%%%%%%%%%%%%%%%%%

{\color {Blue} 

\Problem 5
Verify that the DE is not exact. Find an integration factor $\mu(x)$, such that multiplying by $\mu(x)$ makes the DE exact. You must clearly state how $\mu(x)$ was found. Finally, solve the DE.
$$(\sin(y)-2ye^{-x}\sin(x))dx + (\cos(y)+2e^{-x}\cos(x))dy=0.$$ 

\myLine
}


\begin{Solution}

Consider DE:
$$(\sin(y)-2ye^{-x}\sin(x))dx + (\cos(y)+2e^{-x}\cos(x))dy=0 $$ 

Let:
$$ M = \sin(y)-2ye^{-x}\sin(x) $$
$$ N =  \cos(y)+2e^{-x}\cos(x) $$

To check if the DE is exact:
    $$ M_y = \cos{y} - 2e^{-x}\sin{x} $$
    $$ N_x = 2 (-e^{-x}\cos{x} -e^x sin{x}) $$

Because $ M_y $ not equal $ N_x  $ so the DE is not exact!

Let $\mu(x)$ is the integration factor 

$$\mu = e^{\int \frac{M_y - N_x}{N} dx } = e^x  $$ 

Integration factor $\mu(x) = e^x $ 

Multiply both sides by the integration factor $\mu(x) = e^x $ 
$$ e^x(\sin(y)-2ye^{-x}\sin(x))dx + e^x(\cos(y)+2e^{-x}\cos(x))dy=0.$$ 

Let
$$ M = e^x(\sin(y)-2ye^{-x}\sin(x)) = e^x\sin(y)-2y\sin(x)  $$
$$ N =  e^x(\cos(y)+2e^{-x}\cos(x)) = e^x\cos(y) + 2\cos(x) $$

The solution 
$$ f(x, y) = \int e^x\sin(y)-2y\sin(x) + h(y) $$

$$ f(x, y) = e^x\sin(y) + 2y\cos(x) + h(y) $$

$$ \frac{\partial}{\partial{y}} ( e^x\sin(y) + 2y\cos(x) + h(y)) = e^x cos(y) + 2cos(x) + h'(y)$$

\eject 

Set $ e^x cos(y) + 2cos(x) + h'(y) = N$ gives 

$$ e^x cos(y) + 2cos(x) + h'(y) = e^x\cos(y) + 2\cos(x) $$ 

$$ h'(y) = 0 => h(y) = C $$

so the solution is 

$$ f(x, y) = e^x\sin(y) + 2y\cos(x) = C $$

\end{Solution}
\eject

%%%%%%%%%%%%%%%%%%

\begin{handout}
\bigskip

Additional sheet for Problem \rule{1cm}{0.2pt}

\bigskip

\fillWithLines {26}
\end{handout}

\end{document}
