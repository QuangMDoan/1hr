%\documentclass{article}

\documentclass[11pt]{article}

\usepackage{amsmath}
\usepackage{amsfonts}
\usepackage{amssymb}
\usepackage{mathtools} 
\usepackage{amsthm}
\usepackage[dvipsnames]{xcolor}
\usepackage{setspace}
\usepackage{enumitem}
\usepackage{romannum}
\usepackage{comment}

\usepackage{xfp}
\usepackage{fp}
\usepackage{forloop}
\usepackage{pgffor}

\makeatletter % Use \makeatletter to access internal LaTeX commands

%% Sets margins
\usepackage{geometry}
 \geometry{
 letterpaper,
 total={165mm,257mm},
 left=20mm,
 top=20mm,
 }
 
 

%%NEWENVIRONMENTS:
\newenvironment{ea}{\begin{list}{\alph{ea}.}
	{\usecounter{ea}}}{\end{list}}

%NEWCOUNTERS:
\newcounter{ea}

%NEW COMMANDS
\newcommand {\Problem} [1] { \bigskip \noindent\Large {\bf Problem #1} \\  \normalsize}
\newcommand {\myLine} {\bigskip \noindent\rule{17.7cm}{0.4pt} \medskip}

\newcommand {\mySpaceLeft}{\the\dimexpr\pagegoal-\pagetotal\relax}


\newcommand{\fillSpaceWithLines}[1]{

  \edef\mySpaceLeft{\the\dimexpr\pagegoal-\pagetotal\relax}
  \edef\myReal{\fpeval{\mySpaceLeft/1pt}}
  \edef\myLines{\fpeval{round(\myReal/#1)}}
  \foreach \varLines in {1,...,\myLines} {
    \dotfill
    \vspace{3mm}
    
  }
  \eject
}  

\newcommand{\fillWithLines}[1]{

\bigskip


\bigskip

	\fillSpaceWithLines {#1}
}

\newcommand{\addPage}{

Additional sheet for problem \rule{3cm}{0.2pt}
\bigskip

\fillSpaceWithLines {26}

}

%\excludecomment{latex}
\includecomment{latex}
\excludecomment{handout}
%\includecomment{handout}

\begin{handout}
\newenvironment{Solution}{
	\bigskip

	\fillSpaceWithLines {26}
	}{ }
\end{handout}

\begin{latex}
\newenvironment{Solution}{
	\underline{Solution:}
	}{ }
\end{latex}

	
%% Initial instructions: 
%% Below is the "visible" part of the LaTeX file. Include your solution where it is marked.
%% Note that some *new commands* have been defined above. You may use them instead of the original LaTeX commands. 
%% Suggesting to use the $$...$$ to display centered text of formulas to emphasize your major results.
%% Please keep one problem per page for easier reading.

\begin{document}

\setstretch {1.2}

\begin{center}
  \begin{Large}
    XM531 Problem Set 2
    
    \medskip
	\ 
	\begin{latex}
   	 Your Name
	\end{latex}
  \end{Large} \\
\end{center}

\medskip


%%%%%%%%%%%%%%%%%%

{\color {Blue} 

\Problem 1 If $W(f,g)$ is the Wronskian of $f$ and $g$, and if $u=2f-g$, $v=f+2g$, find the Wronskian of $W(u,v)$ of $u$ and $v$ in terms of $W(f,g)$

\myLine

}

\begin{Solution}

\end{Solution}

\eject

%%%%%%%%%%%%%%%%%%

{\color {Blue} 

\Problem 2 Assume that $y_1$ and $y_2$ are a fundamental aset of solutions of $y''+p(t)y'+q(t)y=0$ and let $y_3=a_1y_1+a_2y_2$ and $y_4=b_1y_1 +b_2y_2$, where $a_1,a_2,b_1,$ and $b_2$ are any constants. Show that
$$W(y_3,y_4) = (a_1b_2 - a_2b_1)W(y_1,y_2).$$
Are $y_3$ and $y_4$ also a fundamental set of solutions? Why or why not?

\myLine
}


\begin{Solution}

\end{Solution}

\eject

%%%%%%%%%%%%%%%%%%

{\color {Blue} 

\Problem 3 Find the fundamental set of solutions specified by Theorem 3.2.5 for the given differential equation and initial point.
$$y'' + y' - 2y = 0; \ \ \ \ t_0 = 0$$

\myLine
}


\begin{Solution}

\end{Solution}

\eject

%%%%%%%%%%%%%%%%%%

{\color {Blue} 
\Problem 4
Find the characteristic polynomial, and then write the general solution to the ODEs.
\begin{enumerate}[label=(\alph*)]
\item  $\displaystyle{y'' - 4y' - 12y=0}$.
\item  $\displaystyle{-\frac{1}{2}y''  = 13y + 5y'}$.
\item  $y''+9y=6y'$.
\end{enumerate}
\myLine
}


\begin{Solution}

\end{Solution}

\eject

%%%%%%%%%%%%%%%%%%

{\color {Blue} 

\Problem 5
A $3$-foot spring measures $9$ feet long after a mass weighing $12$ pounds is attached to it. The medium through which the mass moves offers a damping force numerically equal to $\sqrt{3}$ times the instantaneous velocity. Find the equation of motion if the mass is initially released from the equilibrium position with a downward velocity of $5$ ft/s. Hint: Use $g = 32$ $\textrm{ft}/\textrm{s}^2$ for the acceleration due to gravity.

\myLine
}


\begin{Solution}

\end{Solution}
\eject

%%%%%%%%%%%%%%%%%%

\begin{handout}
\bigskip

Additional sheet for Problem \rule{1cm}{0.2pt}

\bigskip

\fillWithLines {26}
\end{handout}

\end{document}

