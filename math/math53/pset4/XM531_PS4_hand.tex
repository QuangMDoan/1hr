%\documentclass{article}

\documentclass[11pt]{article}

\usepackage{amsmath}
\usepackage{amsfonts}
\usepackage{amssymb}
\usepackage{mathtools} 
\usepackage{amsthm}
\usepackage[dvipsnames]{xcolor}
\usepackage{setspace}
\usepackage{enumitem}
\usepackage{romannum}
\usepackage{comment}
\usepackage{multimedia}

\usepackage{xfp}
\usepackage{fp}
\usepackage{forloop}
\usepackage{pgffor}

\makeatletter % Use \makeatletter to access internal LaTeX commands

%% Sets margins
\usepackage{geometry}
 \geometry{
 letterpaper,
 total={165mm,257mm},
 left=20mm,
 top=20mm,
 }
 
 

%%NEWENVIRONMENTS:
\newenvironment{ea}{\begin{list}{\alph{ea}.}
	{\usecounter{ea}}}{\end{list}}

%NEWCOUNTERS:
\newcounter{ea}

%NEW COMMANDS
\newcommand {\Problem} [1] { \bigskip \noindent\Large {\bf Problem #1} \\  \normalsize}
\newcommand {\myLine} {\bigskip \noindent\rule{17.7cm}{0.4pt} \medskip}

\newcommand {\mySpaceLeft}{\the\dimexpr\pagegoal-\pagetotal\relax}


\newcommand{\fillSpaceWithLines}[1]{

  \edef\mySpaceLeft{\the\dimexpr\pagegoal-\pagetotal\relax}
  \edef\myReal{\fpeval{\mySpaceLeft/1pt}}
  \edef\myLines{\fpeval{round(\myReal/#1)}}
  \foreach \varLines in {1,...,\myLines} {
    \dotfill
    \vspace{3mm}
    
  }
  \eject
}  

\newcommand{\fillWithLines}[1]{

\bigskip


\bigskip

	\fillSpaceWithLines {#1}
}

\newcommand{\addPage}{

Additional sheet for problem \rule{3cm}{0.2pt}
\bigskip

\fillSpaceWithLines {26}

}

%\excludecomment{latex}
\includecomment{latex}
\excludecomment{handout}
%\includecomment{handout}

\begin{handout}
\newenvironment{Solution}{
	\bigskip

	\fillSpaceWithLines {26}
	}{ }
\end{handout}

\begin{latex}
\newenvironment{Solution}{
	\underline{Solution:}
	}{ }
\end{latex}

	
%% Initial instructions: 
%% Below is the "visible" part of the LaTeX file. Include your solution where it is marked.
%% Note that some *new commands* have been defined above. You may use them instead of the original LaTeX commands. 
%% Suggesting to use the $$...$$ to display centered text of formulas to emphasize your major results.
%% Please keep one problem per page for easier reading.

\begin{document}

\setstretch {1.2}

\begin{center}
  \begin{Large}
    XM531 Problem Set 4
    
    \medskip
	\ 
	\begin{latex}
   	 Your Name
	\end{latex}
  \end{Large} \\
\end{center}

\medskip


%%%%%%%%%%%%%%%%%%

{\color {Blue} 

\Problem 1 
The \emph{gamma function} is denoted by $\Gamma(p)$ and is defined by the integral
$$\Gamma(p+1)=\int_0^\infty e^{-x}x^pdx.$$
\begin{ea}
\item Show that, for $p>0, \Gamma(p+1) = p\Gamma(p)$.
\item Show that $\Gamma(1)=1$.
\item If $p$ is a positive integer $n$, show that $\Gamma(n+1) = n!$.
Since $\Gamma(p)$ is also defined when $p$ is not an integer, this function provides an extnsion of the factorial function to nonintegral vales of the independent variable. Note that it is also consistent to define $0!=1$.
\item Show that, for $p>0$, 
$$p(p+1)(p+2)\dots (p+n+1) = \frac{\Gamma(p+n)}{\Gamma(p)}.$$ 
\item Thus $\Gamma(p)$ can be determined for al positive values of $p$ if $\gamma(p)$ is known in a single interval of unit length , say $0<p\leq 1$. It is possible to show that $\Gamma\left(\frac{1}{2}\right)=\sqrt{\pi}$. Find $\Gamma\left(\frac{3}{2}\right)$ and $\Gamma\left(\frac{11}{2}\right)$.
\end{ea}
\myLine

}

\begin{Solution}

\end{Solution}

\eject

%%%%%%%%%%%%%%%%%%

{\color {Blue} 
\Problem 2 
Use the Laplace transform to solve the given intial value problem.
\[y''+3y'+2y= 0;\ \ \ \ \ y(0)=1, \ \ y'(0)=0.\]
\myLine
}


\begin{Solution}

\end{Solution}

\eject

%%%%%%%%%%%%%%%%%%

{\color {Blue} 

\Problem 3 
Let $\alpha$ be a real number. Express the solution of the given initial value problm in terms of a convolution integral.
\[y''+2y'+2y = \sin(\alpha t);\ \ \ \ \ y(0)=0,\ \ y'(0)=0.\]
\myLine
}


\begin{Solution}

\end{Solution}

\eject

%%%%%%%%%%%%%%%%%%

{\color {Blue} 
\Problem 4
Let $\alpha$ be a real number. Find the Laplace transform of the following:

\begin{ea}
\item $f(t) = t^2\sinh(\alpha t)$\\
\item $g(t) = \begin{cases} 
t,& 0\leq t < 1,\\
2-t,& 1\leq t< 2,\text{ and}\\
0,& 2\leq t <\infty
\end{cases}$
\end{ea}
\myLine
}


\begin{Solution}

\end{Solution}

\eject

%%%%%%%%%%%%%%%%%%

\begin{handout}
\bigskip

Additional sheet for Problem \rule{1cm}{0.2pt}

\bigskip

\fillWithLines {26}
\end{handout}

\end{document}

